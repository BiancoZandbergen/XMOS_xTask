\newpage
\section{Application Programming Interface}
In this appendix we will describe the application programming interface that
programmers should use to write applications on systems using the
xTask Operating System.

Applications that make use of the xTask API should include the following 
header file: \verb|/include/xtask.h|.

%-------------------------------------------------------------------------------
%                              xtask_kernel
%-------------------------------------------------------------------------------
\subsection{xtask\_kernel}
\noindent
\textbf{void xtask\_kernel(init\_tasks, idle\_task, tick\_rate,
notification\_chan, service\_chan)}\\\\
Initialize and start a kernel. The kernel is connected to a 
Communication Server through the service and notification channels.
This function should be called in a \verb|par| statement from the main function 
(using a C wrapper function to circumvent xC's dislike of function pointers).\\

\noindent
\textbf{Arguments:}\\
\indent\begin{tabular}{ p{4.5cm}  p{9cm} }
init\_tasks & Pointer to function that initializes all initial tasks.
              This function has the signature: \verb|void function(void)|. \\
idle\_task  & Pointer to idle task function.
              This function has the signature: \verb|void function(void *)|.\\
unsigned int
tick\_rate  & kernel tick rate (by default in 10ns resolution).\\
chanend
notification\_chan & channel for notifications from Communication Server.\\
chanend
service\_chan      & channel to get service from Communication Server.
\end{tabular}\\\\

\noindent
\textbf{Return value:}\\
\indent\begin{tabular}{  p{4.5cm}  p{9cm} }
void & \\
\end{tabular}

%-------------------------------------------------------------------------------
%                              xtask_comserver
%-------------------------------------------------------------------------------
\subsection{xtask\_comserver}
\noindent
\textbf{void xtask\_comserver(service\_chan[], noficication\_chan[], nr\_kernels, 
        ring\_in, ring\_out, id)}\\\\
Initialize and start the communication server.
This function should be called in a \verb|par| statement from the main function.\\

\noindent
\textbf{Arguments:}\\
\indent\begin{tabular}{ p{4.5cm}  p{9cm} }
service\_chan[]      & Array with the service channels connecting 
                       to each kernel.\\
notification\_chan[] & Array with the notification channels connecting
                       to each kernel.\\
unsigned int
nr\_kernels          & Number of kernels connected to this Communication Server.
                       \verb|service_chan[]| and \verb|notification_chan[]|
                       should have \verb|nr_kernels| elements.\\
chanend
ring\_in             & Ring bus incoming chanend. This value can be \verb|null|
                       to disable ring bus. If either \verb|ring_in| 
                       or \verb|ring_out| is \verb|null|,
                       the ring bus will not be enabled.\\
chanend
ring\_in             & Ring bus incoming chanend. This value can be \verb|null|
                       to disable ring bus. If either \verb|ring_in| 
                       or \verb|ring_out| is \verb|null|,
                       the ring bus will not be enabled.\\
unsigned int id      & Globally unique ID for Communication Server.
\end{tabular}\\\\

\noindent
\textbf{Return value:}\\
\indent\begin{tabular}{  p{4.5cm}  p{9cm} }
void & \\
\end{tabular}

%-------------------------------------------------------------------------------
%                              xtask_create_init_task
%-------------------------------------------------------------------------------
\begin{samepage}
\subsection{xtask\_create\_init\_task}
\noindent
\textbf{int xtask\_create\_init\_task(code, stack\_size, priority, tid, args)}\\\\
Create an initial task (before the Operating System starts). One or more of these
function calls should be wrapped in another function and passed to 
\verb|xtask_kernel()|.\\

\noindent
\textbf{Arguments:}\\
\indent\begin{tabular}{ p{4.5cm}  p{9cm} }
code                     & Pointer to the function that should be ran as a task.
                           The function has the following signature: 
                           \verb|void function(void *)|.\\
unsigned int stack\_size & Stack size in 32-bit words.\\
unsigned int priority    & Task priority, a number between 0 and 6. Lower number
                           means higher priority.\\
unsigned int tid         & Unique task ID.\\
void * args              & Arguments passed to the new task (can be NULL).
\end{tabular}\\\\

\noindent
\textbf{Return value:}\\
\indent\begin{tabular}{  p{4.5cm}  p{9cm} }
0 & Always returns 0 currently. \\
\end{tabular}
\end{samepage}

%-------------------------------------------------------------------------------
%                              xtask_create_task
%-------------------------------------------------------------------------------
\begin{samepage}
\subsection{xtask\_create\_task}
\noindent
\textbf{int xtask\_create\_task(code, stack\_size, priority, tid, args)}\\\\
Create a new task by another task.\\

\noindent
\textbf{Arguments:}\\
\indent\begin{tabular}{ p{4.5cm}  p{9cm} }
code                     & Pointer to the function that should be ran as a task.
                           The function has the following signature: 
                           \verb|void function(void *)|.\\
unsigned int stack\_size & Stack size in 32-bit words.\\
unsigned int priority    & Task priority, a number between 0 and 6. Lower number
                           means higher priority.\\
unsigned int tid         & Unique task ID.\\
void * args              & Arguments passed to the new task (can be NULL).
\end{tabular}\\\\

\noindent
\textbf{Return value:}\\
\indent\begin{tabular}{  p{4.5cm}  p{9cm} }
0 & Always returns 0 currently. \\
\end{tabular}
\end{samepage}

%-------------------------------------------------------------------------------
%                              xtask_delay_ticks
%-------------------------------------------------------------------------------
\begin{samepage}
\subsection{xtask\_delay\_ticks}
\noindent
\textbf{void xtask\_delay\_ticks(unsigned int ticks)}\\\\
Delay task for a certain amount of kernel ticks. The task will not be scheduled
while it is delayed, giving other tasks the oportunity to run.\\

\noindent
\textbf{Arguments:}\\
\indent\begin{tabular}{ p{4.5cm}  p{9cm} }
ticks & Number of kernel ticks to delay the task.
\end{tabular}\\\\

\noindent
\textbf{Return value:}\\
\indent\begin{tabular}{  p{4.5cm}  p{9cm} }
void \\
\end{tabular}
\end{samepage}

