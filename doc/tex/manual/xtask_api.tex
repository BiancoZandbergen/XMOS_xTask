\newpage
\section{Application Programming Interface}
In this appendix we will describe the application programming interface that
programmers should use to write applications on systems using the
xTask Operating System.

Applications that make use of the xTask API should include the following 
header file: \verb|/include/xtask.h|.

%-------------------------------------------------------------------------------
%                              xtask_kernel
%-------------------------------------------------------------------------------
\subsection{vc\_buf structure}
\noindent
The vc\_buf structure is used to contain the buffer information for communication
using virtual channels and mailboxes. It contains the following information:
pointer to start of data buffer, data buffer size and the amount of data in the buffer.

\begin{verbatim}
struct vc_buf {
  void *       data;
  unsigned int buf_size;
  unsigned int data_size;
};
\end{verbatim}



%-------------------------------------------------------------------------------
%                              xtask_kernel
%-------------------------------------------------------------------------------
\subsection{xtask\_kernel}
\noindent
\textbf{void xtask\_kernel(init\_tasks, idle\_task, tick\_rate,
notification\_chan, service\_chan)}\\\\
Initialize and start a kernel. The kernel is connected to a 
Communication Server through the service and notification channels.
This function should be called in a \verb|par| statement from the main function 
(using a C wrapper function to circumvent xC's dislike of function pointers).\\

\noindent
\textbf{Arguments:}\\
\indent\begin{tabular}{ p{4.5cm}  p{9cm} }
init\_tasks & Pointer to function that initializes all initial tasks.
              This function has the signature: \verb|void function(void)|. \\
idle\_task  & Pointer to idle task function.
              This function has the signature: \verb|void function(void *)|.\\
unsigned int
tick\_rate  & kernel tick rate (by default in 10ns resolution).\\
chanend
notification\_chan & channel for notifications from Communication Server.\\
chanend
service\_chan      & channel to get service from Communication Server.
\end{tabular}\\\\

\noindent
\textbf{Return value:}\\
\indent\begin{tabular}{  p{13.5cm} }
This function never returns, it will switch to the first task to be ran.
\end{tabular}

%-------------------------------------------------------------------------------
%                              xtask_comserver
%-------------------------------------------------------------------------------
\subsection{xtask\_comserver}
\noindent
\textbf{void xtask\_comserver(service\_chan[], noficication\_chan[], nr\_kernels, 
        ring\_in, ring\_out, id)}\\\\
Initialize and start the communication server.
This function should be called in a \verb|par| statement from the main function.\\

\noindent
\textbf{Arguments:}\\
\indent\begin{tabular}{ p{4.5cm}  p{9cm} }
service\_chan[]      & Array with the service channels connecting 
                       to each kernel.\\
notification\_chan[] & Array with the notification channels connecting
                       to each kernel.\\
unsigned int
nr\_kernels          & Number of kernels connected to this Communication Server.
                       \verb|service_chan[]| and \verb|notification_chan[]|
                       should have \verb|nr_kernels| elements.\\
chanend
ring\_in             & Ring bus incoming chanend. This value can be \verb|null|
                       to disable ring bus. If either \verb|ring_in| 
                       or \verb|ring_out| is \verb|null|,
                       the ring bus will not be enabled.\\
chanend
ring\_in             & Ring bus incoming chanend. This value can be \verb|null|
                       to disable ring bus. If either \verb|ring_in| 
                       or \verb|ring_out| is \verb|null|,
                       the ring bus will not be enabled.\\
unsigned int id      & Globally unique ID for Communication Server.
\end{tabular}\\\\

\noindent
\textbf{Return value:}\\
\indent\begin{tabular}{  p{13.5cm} }
This function never returns, it will process events infinitely.
\end{tabular}

%-------------------------------------------------------------------------------
%                              xtask_create_init_task
%-------------------------------------------------------------------------------
\begin{samepage}
\subsection{xtask\_create\_init\_task}
\noindent
\textbf{int xtask\_create\_init\_task(code, stack\_size, priority, tid, args)}\\\\
Create an initial task (before the Operating System starts). One or more of these
function calls should be wrapped in another function and passed to 
\verb|xtask_kernel()|.\\

\noindent
\textbf{Arguments:}\\
\indent\begin{tabular}{ p{4.5cm}  p{9cm} }
code                     & Pointer to the function that should be ran as a task.
                           The function has the following signature: 
                           \verb|void function(void *)|.\\
unsigned int stack\_size & Stack size in 32-bit words.\\
unsigned int priority    & Task priority, a number between 0 and 6. Lower number
                           means higher priority.\\
unsigned int tid         & Unique task ID.\\
void * args              & Arguments passed to the new task (can be NULL).
\end{tabular}\\\\

\noindent
\textbf{Return value:}\\
\indent\begin{tabular}{  p{4.5cm}  p{9cm} }
0 & Always returns 0 currently. \\
\end{tabular}
\end{samepage}

%-------------------------------------------------------------------------------
%                              xtask_create_task
%-------------------------------------------------------------------------------
\begin{samepage}
\subsection{xtask\_create\_task}
\noindent
\textbf{int xtask\_create\_task(code, stack\_size, priority, tid, args)}\\\\
Create a new task by another task.\\

\noindent
\textbf{Arguments:}\\
\indent\begin{tabular}{ p{4.5cm}  p{9cm} }
code                     & Pointer to the function that should be ran as a task.
                           The function has the following signature: 
                           \verb|void function(void *)|.\\
unsigned int stack\_size & Stack size in 32-bit words.\\
unsigned int priority    & Task priority, a number between 0 and 6. Lower number
                           means higher priority.\\
unsigned int tid         & Unique task ID.\\
void * args              & Arguments passed to the new task (can be NULL).
\end{tabular}\\\\

\noindent
\textbf{Return value:}\\
\indent\begin{tabular}{  p{4.5cm}  p{9cm} }
0 & Always returns 0 currently. \\
\end{tabular}
\end{samepage}

%-------------------------------------------------------------------------------
%                              xtask_delay_ticks
%-------------------------------------------------------------------------------
\begin{samepage}
\subsection{xtask\_delay\_ticks}
\noindent
\textbf{void xtask\_delay\_ticks(unsigned int ticks)}\\\\
Delay task for a certain amount of kernel ticks. The task will not be scheduled
while it is delayed, giving other tasks the oportunity to run.\\

\noindent
\textbf{Arguments:}\\
\indent\begin{tabular}{ p{4.5cm}  p{9cm} }
ticks & Number of kernel ticks to delay the task.
\end{tabular}\\\\

\noindent
\textbf{Return value:}\\
\indent\begin{tabular}{  p{4.5cm}  p{9cm} }
void \\
\end{tabular}
\end{samepage}

%-------------------------------------------------------------------------------
%                              xtask_create_mailbox
%-------------------------------------------------------------------------------
\begin{samepage}
\subsection{xtask\_create\_mailbox}
\noindent
\textbf{unsigned int xtask\_create\_mailbox(id, inbox\_size, outbox\_size)}\\\\
Create a new mailbox for inter-task communication.\\

\noindent
\textbf{Arguments:}\\
\indent\begin{tabular}{ p{4.5cm}  p{9cm} }
unsigned int id          & Globally unique mailbox identifier.\\
unsigned int inbox\_size  & Size of inbox in bytes.            \\
unsigned int outbox\_size & size of outbox in bytes.           
\end{tabular}\\\\

\noindent
\textbf{Return value:}\\
\indent\begin{tabular}{  p{4.5cm}  p{9cm} }
0 & Currently always returns 0. \\
\end{tabular}
\end{samepage}

%-------------------------------------------------------------------------------
%                              xtask_get_outbox
%-------------------------------------------------------------------------------
\begin{samepage}
\subsection{xtask\_get\_outbox}
\noindent
\textbf{struct vc\_buf *  xtask\_get\_outbox(id)}\\\\
Get access to outbox buffer.\\

\noindent
\textbf{Arguments:}\\
\indent\begin{tabular}{ p{4.5cm}  p{9cm} }
unsigned int id          & Globally unique mailbox identifier.           
\end{tabular}\\\\

\noindent
\textbf{Return value:}\\
\indent\begin{tabular}{  p{13.5cm} }
Pointer to vc\_buf structure that contains the buffer information.
\end{tabular}
\end{samepage}

%-------------------------------------------------------------------------------
%                              xtask\_send\_outbox
%-------------------------------------------------------------------------------
\begin{samepage}
\subsection{xtask\_send\_outbox}
\noindent
\textbf{unsigned int xtask\_send\_outbox(sender, recipient)}\\\\
Send outbox to recipient. The sending task will be blocked until the message
is delivered to the recipient task and the recipient task has actively received it.\\

\noindent
\textbf{Arguments:}\\
\indent\begin{tabular}{ p{4.5cm}  p{9cm} }
unsigned int sender    & Globally unique mailbox identifier of sender.\\
unsigned int recipient & Globally unique mailbox identifier of recipient.\\          
\end{tabular}\\\\

\noindent
\textbf{Return value:}\\
\indent\begin{tabular}{  p{4.5cm}  p{9cm} }
0 & Message delivered. \\
1 & Recipient could not be found. 
\end{tabular}
\end{samepage}

%-------------------------------------------------------------------------------
%                              xtask_get_inbox
%-------------------------------------------------------------------------------
\begin{samepage}
\subsection{xtask\_get\_outbox}
\noindent
\textbf{struct vc\_buf * xtask\_get\_inbox(id, location)}\\\\
Receive a message. The task will be blocked if no message is available.\\

\noindent
\textbf{Arguments:}\\
\indent\begin{tabular}{ p{4.5cm}  p{9cm} }
unsigned int id          & Globally unique mailbox identifier. \\
unsigned int location    & Location can be \verb|LOCAL_TILE| or \verb|ALL_TILES|.
                           If one or more tasks have tried to send a message 
                           while the recipient task was not waiting for it, 
                           a flag will be  set in the recipient task's mailbox 
                           to indicate that another task tried to send a message.
                           When the recipient task calls this function it will 
                           inform all tasks that tried to send a message that it
                           is now ready to receive a message. If \verb|LOCAL_TILE|
                           is given as argument it will only search for pending
                           senders that make use of the same Communication Server.
                           If \verb|ALL_TILES| is given, the ring bus will be used
                           to inform all Communication Servers. If all possible
                           sending tasks make use of the same Communication Server,
                           \verb|LOCAL_TILE| should be used.           
\end{tabular}\\\\

\noindent
\textbf{Return value:}\\
\indent\begin{tabular}{  p{13.5cm} }
Pointer to vc\_buf structure that contains the
buffer information of the received message.
\end{tabular}
\end{samepage}

%-------------------------------------------------------------------------------
%                              xtask_create_thread
%-------------------------------------------------------------------------------
\begin{samepage}
\subsection{xtask\_create\_thread}
\noindent
\textbf{unsigned int xtask\_create\_thread(code, stackwords, args, obj\_size, rx\_buf\_size, tx\_buf\_size)}\\\\
Create a new dedicated hardware thread (local, same tile).\\

\noindent
\textbf{Arguments:}\\
\indent\begin{tabular}{ p{4.5cm}  p{9cm} }
code                        & Pointer to the function that should be ran as a
                              dedicated hardware thread.
                              The function has the following signature: 
                              \verb|void function(void *, chanend)|.\\
unsigned int stackwords     & Stack size in 4-byte words.\\
void * args                 & Arguments passed to the new task (can be NULL).\\
unsigned int obj\_size      & Size in bytes of objects transferred through 
                              the channel.
                              Must be a multiple of 4 bytes.\\
unsigned int rx\_buf\_size  & Receive buffer size. Must be a multiple of obj\_size.\\
unsigned int tx\_buf\_size  & Transfer buffer size. Must be a multiple of obj\_size.       
\end{tabular}\\\\

\noindent
\textbf{Return value:}\\
\indent\begin{tabular}{  p{13.5cm} }
A handle to the new dedicated hardware thread.
\end{tabular}
\end{samepage}

%-------------------------------------------------------------------------------
%                              xtask_create_remote_thread
%-------------------------------------------------------------------------------
\begin{samepage}
\subsection{xtask\_create\_remote\_thread}
\noindent
\textbf{unsigned int xtask\_create\_remote\_thread(code, stackwords, args, obj\_size, rx\_buf\_size, tx\_buf\_size)}\\\\
Create a new dedicated hardware thread (different tile).\\
\textbf{This function is highly expirimental!}\\

\noindent
\textbf{Arguments:}\\
\indent\begin{tabular}{ p{4.5cm}  p{9cm} }
unsigned int code           & Function number to execute (not a function pointer!).
                              The function has the following signature: 
                              \verb|void function(void *, chanend)|.\\
unsigned int stackwords     & Stack size in 4-byte words.\\
void * args                 & Arguments passed to the new task (can be NULL).\\
unsigned int obj\_size      & Size in bytes of objects transferred through 
                              the channel.
                              Must be a multiple of 4 bytes.\\
unsigned int rx\_buf\_size  & Receive buffer size. Must be a multiple of obj\_size.\\
unsigned int tx\_buf\_size  & Transfer buffer size. Must be a multiple of obj\_size.       
\end{tabular}\\\\

\noindent
\textbf{Return value:}\\
\indent\begin{tabular}{  p{13.5cm} }
A handle to the new dedicated hardware thread.
\end{tabular}
\end{samepage}

%-------------------------------------------------------------------------------
%                              xtask_vc_get_write_buf
%-------------------------------------------------------------------------------
\begin{samepage}
\subsection{xtask\_vc\_get\_write\_buf}
\noindent
\textbf{struct vc\_buf * xtask\_vc\_get\_write\_buf(handle)}\\\\
Receive a write buffer that can be filled by the task and
transmitted to the dedicated hardware thread. This function
should only be called once prior to the first transmission
to the dedicated hardware thread. The function that sends
the buffer to the dedicated hardware thread will return a
new empty buffer.\\

\noindent
\textbf{Arguments:}\\
\indent\begin{tabular}{ p{4.5cm}  p{9cm} }
unsigned int handle      & Dedicated hardware thread handle. \\           
\end{tabular}\\\\

\noindent
\textbf{Return value:}\\
\indent\begin{tabular}{  p{13.5cm} }
A pointer to a vc\_buf struct that contains the information
about the buffer that can be filled by the task and
transmitted to the dedicated hardware thread.
Null pointer when no buffer was available but this should
not happen in regular operation.
\end{tabular}
\end{samepage}

%-------------------------------------------------------------------------------
%                              xtask_vc_receive
%-------------------------------------------------------------------------------
\begin{samepage}
\subsection{xtask\_vc\_receive}
\noindent
\textbf{struct vc\_buf * xtask\_vc\_receive(handle, min\_size)}\\\\
Receive from a hardware thread through a virtual channel.
The task will block if there is no (sufficient) data available.\\

\noindent
\textbf{Arguments:}\\
\indent\begin{tabular}{ p{4.5cm}  p{9cm} }
unsigned int handle      & Dedicated hardware thread handle. \\ 
unsigned int min\_size   & Minimum amount of data to receive in
                           bytes. If set to 0, the minimum amount
                           is a full buffer.           
\end{tabular}\\\\

\noindent
\textbf{Return value:}\\
\indent\begin{tabular}{  p{13.5cm} }
Pointer to a vc\_buf struct which contains all the
information about the buffer such as the actual pointer to
the buffer and the amount of data in the buffer.
\end{tabular}
\end{samepage}

%-------------------------------------------------------------------------------
%                              xtask_vc_send
%-------------------------------------------------------------------------------
\begin{samepage}
\subsection{xtask\_vc\_send}
\noindent
\textbf{struct vc\_buf * xtask\_vc\_send(buf)}\\\\
Instruct the Communication Server to send the write buffer
to the dedicated hardware thread. Receive a new empty
write buffer that can be immediately filled by the task.\\

\noindent
\textbf{Arguments:}\\
\indent\begin{tabular}{ p{4.5cm}  p{9cm} }
struct vc\_buf * buf      & Pointer to write buffer.\\            
\end{tabular}\\\\

\noindent
\textbf{Return value:}\\
\indent\begin{tabular}{  p{13.5cm} }
A pointer to a vc\_buf struct that contains the information
about the buffer that can be filled by the task and
transmitted to the dedicated hardware thread.
Null pointer when no buffer was available but this should
not happen in regular operation.
\end{tabular}
\end{samepage}

%-------------------------------------------------------------------------------
%                              xtask_exit
%-------------------------------------------------------------------------------
\begin{samepage}
\subsection{xtask\_exit}
\noindent
\textbf{void xtask\_exit(status)}\\\\
Exit task.\\

\noindent
\textbf{Arguments:}\\
\indent\begin{tabular}{ p{4.5cm}  p{9cm} }
unsigned int status      & Exit status code. Reserved for future use.\\            
\end{tabular}\\\\

\noindent
\textbf{Return value:}\\
\indent\begin{tabular}{  p{13.5cm} }
This function does not return. The task will be halted and its resources freed.
The scheduler will pick the next task to run.
\end{tabular}
\end{samepage}

